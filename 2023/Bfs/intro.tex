% !TeX spellcheck = ru_RU
% !TEX root = bfs_lodygin.tex

\section*{Введение}
\thispagestyle{withCompileDate}

%В настоящее время одним из наиболее востребовательных методов для абстракции реальных систем является %представление этой системы в виде графа.
В настоящее время использование такой структуры данных как \textit{граф} имеет широкое применение для абстракции реальных систем с целью их дальнейшего анализа и обработки. Граф представляет из себя совокупность двух множеств --- объектов, называемых вершинами графа, и рёбер, обозначающих попарные связи между объектами. Такое представление используется в социальных науках, физике, химии, но чаще всего в информатике и сетевых технологиях, например, для хранения моделей машинного обучения, для работы с картами или же для хранения и обработки баз данных.

Для хранения графа в памяти используются различные представления, зависящие от соотношения количества вершин и рёбер в графе. К примеру, граф с относительно большим количеством рёбер часто представляется как \textit{матрица смежности}, где каждый столбец и строка обозначают вершину, а в ячейках хранится информация о наличии связи между данными вершинами. Становится очевидно, что в случае малого количества связей между вершинами, то есть когда исследуется \textit{разреженный граф}, такой подход для хранения не является рациональным, поэтому для оптимизации занимаемой памяти граф можно представить в виде \textit{дерева квадрантов}.

Графовая модель позволяет решать внушительный спектр задач, связанный с отношением между вершинами в графе. Для этого разработано большое количество алгоритмов, одним из которых является алгоритм обхода графа в ширину, иначе \textit{Breadth-first search}. При реализации такого алгоритма возможно применение \textit{векторно-матричных} операций, что позволяет естественным образом использовать \textit{многопоточное программирование} для оптимизации скорости работы алгоритма.

Общее количество вершин и \textit{плотность графа} непосредственно влияют на скорость работы алгоритма обхода в ширину, поэтому важной задачей является не только разработка параллельной версии такого алгоритма, но и исследование влияния этих параметров на его работу. 
