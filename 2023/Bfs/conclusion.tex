% !TeX spellcheck = ru_RU
% !TEX root = bfs_lodygin.tex

\section*{Заключение}
В рамках проведения работы были получены следующие результаты.

\begin{itemize}
\item Реазлизованы типы \texttt{SparseMatrix} и \texttt{SparseVector}.
\item Реализованы параллельные версии векторно-матричных операций --- \texttt{ParallelMultiplyVecMat} и \texttt{ParallelFAddVector}.
\item Реализована параллельная версия обхода графа в ширину.
\item Проведено экспериментальное исследование реализованного алгоритма. Параллельная версия работает быстрее последовательной на любых графах с количеством вершин больше 10. Наибольшую выгоду для производительности при обработке графов относительно крупных размеров даёт использование количества подзадач, равное удвоенному числу логических процессоров в системе.
\end{itemize}
\noindent В качестве будущих задач для исследований можно выделить следующие пункты.
\begin{itemize}
\item Исследование производительности алгоритма обхода графа в ширину в зависимости от типа данных, содержащихся на рёбрах.
\item Выявление оптимального количества подзадач при многопоточном программировании в зависимости от количества оперативной памяти системы.
\end{itemize}

