% !TeX spellcheck = ru_RU
% !TEX root = bfs_lodygin.tex

\section{Постановка задачи}
\label{sec:task}
Целью данной работы является исследование производительности алгоритма обхода графа в ширину в зависимости от входных параметров графа и количества параллельных потоков, используемых для работы алгоритма. Для её выполнения были поставлены следующие задачи:

 \begin{enumerate}
 \item  реализовать типы \texttt{SparseMatrix} и \texttt{SparseVector};
 \item  реализовать параллельные версии векторно-матричных операций;
 \item  реализовать параллельную версию алгоритма обхода графа в ширину;
 \item  провести эксперименты над графами с различными параметрами и ответить на следующие вопросы:
   \begin{itemize}
   \item  при каких параметрах графа выгоднее использовать параллельную версию алгоритма, а при каких последовательную;
   \item  использование какого количества потоков даёт наибольший выигрыш в производительности и почему.
   \end{itemize}
 \end{enumerate}
