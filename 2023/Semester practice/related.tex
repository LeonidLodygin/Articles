% !TeX spellcheck = ru_RU
% !TEX root = game_lodygin.tex

\section{Обзор}
\label{sec:relatedworks}

\subsection{Терминология}
\textit{Игровой движок}~--- программное обеспечение, предназначенное для разработки и создания видеоигр. Он предоставляет набор инструментов и функций, которые облегчают процесс разработки игр, включая управление графикой, звуком, физикой, искусственным интеллектом, вводом пользователя и многими другими аспектами игрового процесса.

\textit{Ассет}~--- любой ресурс, который можно использовать в проекте. Это может включать в себя изображения, звуки, 3D-модели, текстуры, анимации, скрипты, материалы, шейдеры и многое другое. В общем смысле, ассеты представляют собой файлы и ресурсы, которые нужны для создания и развития игры или приложения.

\textit{Префаб}~--- шаблон объекта, который вы можно создать, настроить и затем использовать многократно в разных частях игры без необходимости создавать его снова.

\textit{Компонента}~--- модульный блок функциональности, который можно добавить к игровым объектам для определения их поведения или характеристик.

\textit{Рейкаст (\texttt{Raycast})}~--- техника, которая используется для определения того, на что направлен луч в трехмерном или двухмерном пространстве. Рейкаст создает луч и проверяет, пересекается ли он с каким-либо объектом в сцене. Эта техника часто применяется для решения различных задач, таких как обработка пользовательских взаимодействий, определение столкновений, определение видимости объектов и т.д.

\textit{Скелет (\texttt{Rig})}~--- система костей и контрольных объектов, которые привязываются к 3D-модели, чтобы управлять её движением и анимацией.

\textit{Коллайдер}~--- компонента, которая используется для определения границ и обнаружения столкновений между объектами в трехмерном или двухмерном пространстве. Коллайдеры играют важную роль в физическом взаимодействии объектов.


\subsection{Обзор аналогов}
В качестве аналогов выбирались игры данного жанра со схожим концептом. Обзор таких игр помог выявить интересные механики и удачные решения в визуальной составляющей.

\subsubsection{Resident Evil 7: Biohazard}

Компьютерная игра в жанре \texttt{survival horror}, разработанная и изданная японской компанией \texttt{Capcom}. Особенности игры включают в себя:
\begin{enumerate}
\item  перспективу от первого лица. В отличие от предыдущих частей, где камера была установлена в третьем лице, \texttt{Resident Evil 7}~\cite{enwiki:1188461634} предлагает перспективу от первого лица, что придает игроку более сильное пугающее ощущение;
\item  красивую и продуманную визуальную составляющую, например, покачивание оружия в руке главного героя;
\item  фокус на выживании~--- игроку приходится исследовать поместье, решать головоломки и бороться за выживание.
\end{enumerate}

\subsubsection{Call of Cthulhu: Dark Corners of the Earth}

\texttt{Survival horror} игра~\cite{enwiki:1182890909}, разработанная \texttt{Headfirst} \texttt{Productions}, изданная \texttt{Bethesda Softworks}. Основанная на произведениях Г.Ф. Лавкрафта, особенно на его романе "Тень над Иннсмутом" (\texttt{The Shadow over Innsmouth}), игра предлагает игрокам погрузиться в мрачный и атмосферный мир Лавкрафтовского ужаса. Основные черты игры:
\begin{enumerate}
\item  смешение стилей игры. В игре присутствуют элементы шутера от первого лица, детективной работы и выживания. Игрокам предстоит исследовать окружающий мир, решать головоломки и бороться за выживание в мире, наполненном тайнами и ужасами;
\item  система уровня страха. В игре присутствует уникальная система уровня страха, которая влияет на поведение персонажа в зависимости от того, насколько страшным было прошлое событие;
\item  сюжет и атмосфера. Игра рассказывает историю детектива Джека Уолтерса, который расследует загадочные события в небольшом городе \texttt{Innsmouth}. Сюжет тесно связан с мифологией Ктулху и других творений Лавкрафта, предоставляя игрокам уникальную возможность погрузиться в мир космического ужаса.
\end{enumerate}

\subsection{Существующие технологии}

Для того, чтобы каждый раз не воссоздавать физическое моделирование, визуализацию, взаимодействие между объектами и многие другие аспекты, присущие всем играм, были разработаны игровые движки, позволяющие сосредоточить свое внимание на реализации уже конкретных механик вашей игры. Самыми популярными среди движков, поддерживающих разработку 3D игр являются \texttt{Unity Engine}~\cite{enwiki:1186824491} и \texttt{Unreal Engine}~\cite{unrealengine}.

\subsubsection{Unity Engine}

\texttt{Unity}~--- кроссплатформенный игровой движок, который разрабатывается и поддерживается компанией \texttt{Unity Technologies}. Этот движок используется для создания различных видеоигр, виртуальной реальности, анимаций, симуляций и других интерактивных контентов. Имеет активное сообщество и обширную библиотеку ассетов, а также учебные материалы и форум. Основным языком для написания скриптов является \csharp. Концепт движка состоит в том, что каждый объект на сцене является игровым объектом, содержащим различные компоненты, определяющие его поведение и внешний вид.

\subsubsection{Unreal Engine}

\texttt{Unreal Engine}~--- мощный и широко используемый игровой движок, разработанный компанией Epic Games. Он предоставляет разработчикам среду для создания высококачественных 2D и 3D видеоигр, виртуальной реальности, архитектурных визуализаций, тренировочных симуляторов и многого другого. Поддерживает программирование на языке C++ и систему программирования под названием \texttt{Blueprints}, предоставляющую визуальный интерфейс для создания логики игры без необходимости программирования на языке кода.

\subsection{Использованные технологии}

Для разработки данной игры был сделан выбор в пользу \texttt{Unity Engine}, так как имеется опыт разработки на \texttt{.NET}. Также большое количество обучающего материала облегчает процесс изучения возможностей данного движка.

Для создания 3D модели персонажа использовалось приложение \texttt{Fuse}~\cite{Fuse} от компании \texttt{Adobe}, предоставляющее широкий набор гуманоидных моделей. Анимирование модели происходило с помощью связанной со \texttt{Fuse} технологией \texttt{Mixamo}, позволяющей для созданной модели подобрать необходимые костевые анимации (\texttt{rig animation}). Для обрезания рук модели игрока была использована программа \texttt{Blender}~\cite{Blender}.

Для считывания ввода игрока задействован пакет \texttt{Input System}~\cite{InputSystem}, предоставляющий единый интерфейс для работы с различными устройствами ввода.

Для построения окружения был задействован пакет \texttt{ProBuilder}~\cite{ProBuilder}, позволяющий без глубоких знаний в моделировании объемных объектов создавать и гибко настраивать 3D модели.

Для реализации визуальных эффетков были использованы пакеты \texttt{Post Processing}~\cite{PostProcessing} и \texttt{Animation Rigging}~\cite{AnimationRigging}. \texttt{Post Processing} позволяет с помощью компонент на основной камере накладывать необходимые эффекты и фильтры на изображение. \texttt{Animation Rigging} предоставляет интерфейс для создания сложных и интересных анимаций персонажей, в том числе с использованием инверсной кинематики.



%В обзоре необходимо ссылаться на работы других людей. В данном шаблоне задумано, что литература будет указываться в файле \verb=vkr.bib=. В нём указываются пункты литературы в формате \BibTeX{}, а затем на них можно ссылаться с помощью \verb=\cite{...}=. Та литература, на которую Вы сошлетесь, попадет в список литературы в конце документа. Если не сошлетесь~---  не попадёт. Спецификацию в формате \BibTeX{} почти никогда (для второго курса~--- никогда), не нужно придумывать руками. Правильно: находить в интернете описание цитируемой статьи\footnote{Например, \url{https://dl.acm.org/doi/10.1145/3408995} (дата доступа:   \DTMdate{2022-12-17}).},
%копировать цитату с помощью кнопки \foreignquote{english}{Export Citation} и вставлять в \BibTeX{} файл. Если так не делать, но оформление литературы будет обрастать багами.
%Например, \BibTeX{} по особенному обрабатывает точ\-ки, запятые и \verb=and= в списке авторов, что позволяет ему самому понимать, сколько авторов у статьи, и что там фамилия, что~--- имя, а что~--- отчество.

%В обзоре и в остальном тексте вы наверняка будете использовать названия продуктов или языков программирования. Для них рекоменду\-ется (в файле \verb=preamble2.tex=) за\-дать специальные команды, чтобы писать сложные названия правильно и одинаково по всему доку\-менту. Написать с ошибкой  название любимого языка программирова\-ния науч\-ного руко\-водителя~--- идеальный вариант его выбесить.
